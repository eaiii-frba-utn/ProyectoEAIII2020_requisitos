\section{Radio definida por software}

Radio definida por software o SDR (del inglés Software Defined Radio) es un sistema de radiocomunicaciones donde varios de los componentes, típicamente implementados en hardware (mezcladores, filtros, moduladores/demoduladores, detectores, etc), son implementados en software, utilizando computadora personal u otros dispositivos de procesamiento digital de señales. 


Los sistemas de comunicación tiene como objetivo transmitir un mensaje $m(t)$ a través de un canal de banda de paso, es decir, un canal donde solo se puede usar un rango limitado de frecuencias. 
Un buen ejemplo son las transmisiones de radio FM comerciales, generalmente restringidas a una banda de frecuencia entre \SI{85} y \SI{108}{MHz} que debe acomodar múltiples estaciones, a cada una de las cuales se le asigna una banda $\le$ \SI{200}{KHz}. 
La señal debe cambiarse frecuencia para satisfacer los requerimientos del canal de comunicación.
Esto se logra mediante la modulación, por lo que una característica de una onda portadora (generalmente sinusoide como $ A_p  cos(2 \pi f_c t) $ )  variará de acuerdo con la señal de modulación: $ m (t) $ (también llamado el mensaje) que produce una señal modulada : $ s(t) = B(t) \cdot cos(2 \pi f_c  t + \theta(t) ) $. 


Principalmente, hay dos formas  de modulación analógica: 
\begin{itemize}
    \item Modulación de amplitud (AM): donde la característica que se hace para variar es la amplitud B(t)B (t) de la portadora
    \item Modulación angular (PM / FM): donde la característica que varía es la fase $\theta(t)$ de la portadora.
\end{itemize}

Términos comúnmente utilizados en telecomunicaciones:

\begin{itemize}
    \item Banda base (señal): señal que tiene soporte en un rango estrecho de frecuencias alrededor de 0 Hz.
    \item Banda de paso (señal): señal que tiene soporte en un rango estrecho de frecuencias sobre una frecuencia central fc.
    \item Ancho de banda de(de una señal): alguna medida del soporte de una señal en el dominio de la frecuencia (tiene muchas definiciones técnicas diferentes).
\end{itemize}  


Algunos ejemplos son muchos receptores de radio AM/FM convencionales en los que el usuario selecciona un canal sintonizando la radio a su frecuencia de onda portadora.

En un esquema de conversión directa se debe aplicar un filtro lo suficientemente selectivo a la señal que proviene de la antena para filtrar otras señales y ruidos fuera de la banda de interés antes de la de-modulación. 
Para el propósito descrito anteriormente, también debe tener una frecuencia central ajustable que hace que su implementación práctica sea problemática. 
Las arquitecturas superheterodinas resuelven esto convirtiendo primero a una frecuencia intermedia (IF) en la que se pueden aplicar etapas de filtrado y amplificación más rigurosas ahora que la señal está en una frecuencia preestablecida fija.

Esta conversión descendente se logra mezclando con una onda sinusoidal de la frecuencia apropiada generada por un oscilador local (LO). 
Cuando el usuario selecciona una frecuencia $f_c$ para sintonizar, el LO genera una onda sinusoidal en $f_{LO} =f_c -f_{IF}$  (inyección en el lado bajo) o $f_{LO} =f_c  + f_{IF}$  (alto) inyección lateral, tenga en cuenta que cuando utilice esta frecuencia, el espectro de la señal resultante se invertirá en frecuencia). Un subproducto de esta mezcla es que tanto $fc$ como $fc\pm 2 f_{IF}$ (para inyección lateral alta / baja respectivamente) se mezclan en la frecuencia intermedia. La primera es la frecuencia de interés y la otra es la llamada frecuencia de imagen.
  

Por lo tanto, una etapa inicial de filtrado de radiofrecuencia (RF) es útil para filtrar cualquier señal o ruido en frecuencia de imagen. 
Este filtro de RF a menudo tiene una frecuencia central variable cuya sintonía se ajusta con el LO. 
Otro componente común de la sección de RF del receptor es un amplificador de bajo ruido (LNA).
Si bien tradicionalmente la sección de procesamiento de señal de frecuencia intermedia era analógica, últimamente, debido a la utilización de los circuitos integrados y la disponibilidad de microprocesadores en muchos dispositivos (como teléfonos celulares), la tendencia ha sido manejar algunas de estas tareas digitalmente. 
En este caso, las arquitecturas superheterodinas son útiles ya que convierten una señal de banda de paso que es demasiado poco práctica para muestrear (debido a que su alta frecuencia requiere frecuencias de muestreo muy altas) a una señal de banda de paso de frecuencia más baja que es más manejable muestrear sin alias.