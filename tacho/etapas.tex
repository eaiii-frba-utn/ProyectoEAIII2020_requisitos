\section{Diseño de un receptor}

\subsection{Diagrama en bloques}

En la figura se muestra el diagrama en bloques propuesto para el proyecto. 

\input{circuits/diagrama1.tex}


\subsection{Etapa RF}

Esta etapa tiene como principales objetivo la atenación de frecuencia imagen.


\subsection{Diseño del filtrado en el Diagrama en Bloques.} 

A partir del diseño de un Rx Superheterodino clásico se definirá el formato de los filtros de RF, 1 FI y 2 FI, especificando los tres filtro.
Para esta entrega los mezcladores se supondran ideales. 

En ésta entrega deberá estar definido cuantos circuitos  Simple sintonizados y Doble Sintonizados se utilizarán para cumplir los pasabandas establecidos, como así también la frecuencia central y el Q de trabajo de los mismos.

En cada cáso deberá estar claramente definido la topología de filtros, como así también, la frecuencia central , rango de variación de la frecuencia central, ancho de banda, banda de rechazo y planicidad  en banda de paso.



\subsection{Etapa 1 FI}

\subsection{Etapa 2 FI}


SIMULACIÓN.
Se ensayará por simulación la etapa diseñada en la cuarta entrega (Diseño de seña). Explicando las diferencias encontradas entre la simulación y el diseño e introduciendo las modificaciones que pudieran corresponder

Fecha Límite 31 de JULIO 2017 (Curso Lunes) y 10 de AGOSTO 2017 (Curso Jueves)

5.6. Sexta entrega
      AMPLIFICADORES DISEÑO REAL. Se completará el diseño realizado en la cuarta entrega. Para ello se estudiará la estabilidad del amplificador considerando todos los parámetros del elemento activo y  se neutralizará en caso de necesidad;  completando el diseño con los circuitos de polarización.
Completado el diseño real se calculará la cifra de ruido de la etapa 
.
Fecha Límite 14 de AGOSTO 2017 (Curso Lunes) y 24 de AGOSTO 2017 (Curso Jueves)

5.7. Séptima entrega
SIMULACIÓN. Se ensayará por simulación la etapa diseñada en la sexta entrega (Diseño real). Explicando las diferencias encontradas entre la simulación y el diseño e introduciendo las modificaciones que pudieran corresponder

Fecha Límite 18 de SETIEMBRE 2017 (Curso Lunes) y 28 de SETIEMBRE 2017 (Curso Jueves)

5.8. Octava entrega
ARMADO DE ETAPA.  Para ésta instancia se deberán seleccionar y conseguir los componentes, dibujar y realizar la placa de circuito impreso y finalmente armar el conjunto

Fecha Límite 23  de OCTUBRE  2017 (Curso Lunes) y 02 de NOVIEMBRE  2017 (Curso Jueves)

5.9. Novena Entrega
MEDICIONES.  Se deberá medir el prototipo realizado para la octava entrega y realizar el informe final del proyecto con las conclusiones del mismo

Fecha Límite 06  deNOVIEMBRE  2017 (Curso Lunes) y 16 de NOVIEMBRE  2017 (Curso Jueves)






