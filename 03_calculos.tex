\section{Cálculos y verificaciones}

\subsection{Ecuaciones}
Las ecuaciones se puede poner en la linea, $|\bar{A}| = \frac{1}{2}$. También se puede poner centrada como,

$$|\bar{A}| = \frac{1}{\sqrt{1+Q^2\cdot (\frac{f_x}{f_o} - \frac{f_o}{f_x})^2}}$$

Para mas información, se puede consultar el sitio web ``https://ondiz.github.io/cursoLatex'' \footnote{https://ondiz.github.io/cursoLatex/Contenido/05.Ecuaciones.html}.

Se puede usar un editor online como ``https://www.codecogs.com/latex/eqneditor.php''

\subsection{Tablas}
Las tablas se pueden armar online \emph{https://www.tablesgenerator.com/}

\begin{table}[h!]
\centering
\begin{tabular}{|l|l|}
\hline
Frecuencia             & $y_{ie}$          \\ \hline
\SI{455}{\kilo\hertz}  & $0.5 + j\, 0.05$  \\ \hline
\SI{10,7}{\mega\hertz} & $0.5 + j\, 0.05$  \\ \hline
\end{tabular}
\end{table}

\subsection{Template para el diseño del layout.}

CircuitMaker tiene filosofía de diseño basado en la comunidad, con componentes y proyectos de diseño gestionados dentro de una Comunidad dedicada y basada en la nube. En el verdadero espíritu del código abierto, los componentes y diseños pueden ser compartidos libremente entre otros usuarios.

Para poder compartir el proceso de fabricación, los circuitos de PCB deben respetar el diseño propuesto en:

https://workspace.circuitmaker.com/Projects/Details/AlejandroAlm/EAIII2020Template

