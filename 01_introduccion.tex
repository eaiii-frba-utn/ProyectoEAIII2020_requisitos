\section{Introducción}

El presente documento corresponde a la introducción al trabajo práctico anual de Electrónica Aplicada III.   


Este proyecto reviste carácter individual. 
No obstante se acepta que el mismo sea llevado a cabo en forma grupal, teniendo en cuenta que todos los integrantes son responsables de la realización y que deben conocer todos los aspectos de diseño, especificaciones y el estado del mismo. 

Las entregas, en todos los casos, deben ser las mismas para todos los integrantes del grupo. 
No se aceptan entregas distintas para los diferentes integrantes de un grupo. 
La lista de autores debe ser incluida en cada entrega y ordenada de forma alfabética. 

Cada uno de los informes deben ser entregados antes de la fecha de primer parcial, para el informe de funcionamiento, y segundo parcial, para el informe práctico.  

El trabajo será evaluado mediante dos etapas de presentaciones y una presentación oral, donde se desarrollará la teoría y las consideraciones prácticas mediante una presentación. 
Cada etapa debe ser aprobada para tener la aprobación del proyecto. 


\subsection{Informe de funcionamiento}

La presentación del informe de funcionamiento tiene como objetivo 
dar a conocer, en base a fundamentos teóricos, el funcionamiento del circuito. 

En el informe debe explicar el funcionamiento del circuito y cuales pueden ser sus aplicaciones. Además, se deben justificar la topología y componentes del circuito en base a cálculos y simulaciones. 


El informe de funcionamiento debe estar estructurado con el siguiente formato:

\begin{itemize}
    \item Hoja presentación: Título, autores.
    \item Índice
    \item Introducción
    \item Fundamentos teóricos del proyecto
    \item Cálculos y verificaciones.
    \item Conclusiones
    \item Bibliografía
\end{itemize}


\subsection{Informe práctico}

Se requiere la presentación de un informe práctico del trabajo.
Este debe incluir el listado de componentes (incluyendo proveedores y precios), diseño de la placa electrónica y características constructivas. 
Además, se debe detallar que mediciones o pruebas deben ser realizadas para constatar el funcionamiento del circuito. 
La fecha límite de entrega será publicada en el Campus Virtual del curso. Solo serán aceptados los informes finales de los alumnos que tengan aprobado el informe de funcionamiento. 

El informe final debe estar estructurado con el siguiente formato:

\begin{itemize}
    \item Hoja presentación: Titulo, autores.
    \item Índice
    \item Resumen
    \item Introducción
    \item Diseño y consideraciones prácticas
    \item Mediciones y verificaciones.
    \item Conclusiones
    \item Bibliografía
\end{itemize}

\subsection{Presentación oral}

La presentación oral tiene como objetivo la comunicación de los aspectos teóricos y prácticos desarrollados durante el proyecto. 

La presentación oral debe ser realizada con el soporte de un documento tipo presentación que debe estar organizado de la siguiente manera:

\begin{itemize}
    \item Hoja presentación: Titulo, autores.
    \item Descripción del proyecto y usos comunes. 
    \item Fundamentos teóricos del proyecto
    \item Cálculos y verificaciones
    \item Diseño y consideraciones prácticas
    \item Mediciones y verificaciones.
    \item Conclusiones
\end{itemize}